%
% ********** P�gina de Rosto
%

% titlepage gera p�ginas sem numera��o
\begin{titlepage}

\begin{center}

\small

% O comando @{} no ambiente tabular x � para criar um novo delimitador
% entre colunas que n�o a barra vertical | que � normalmente utilizada.
% O delimitador desejado vai entre as chaves. No exemplo, n�o h� nada,
% de modo que o delimitador � vazio. Este recurso est� sendo usado para
% eliminar o espa�o que geralmente existe entre as colunas
\begin{tabularx}{\linewidth}{ c X }
% A figura foi colocada dentro de um parbox para que fique verticalmente
% centralizada em rela��o ao resto da linha
\parbox[c]{3cm}{\includegraphics[width=\linewidth]{IFRN}} &
\begin{center}
\textsf{\textsc{Instituto Federal de Educa��o Ci�ncia\\
 e Tecnologia do Rio Grande do Norte \\
Diretoria de Gest�o de Tecnologia da Informa��o \\
Curso Superior de Tecnologia em Redes de Computadores
}} 
\end{center}

\end{tabularx}


% O vfill � um espa�o vertical que assume a m�xima dimens�o poss�vel
% Os vfill's desta p�gina foram utilizados para que o texto ocupe
% toda a folha
\vfill

\LARGE

\textbf{Aqui � o t�tulo de seu trabalho}

\vfill

\Large

\textbf{Fulano do Anz�is Pereira}

\vfill

\normalsize

Prof. Dr. Rodrigo Siqueira Martins

\vfill

\hfill
\parbox{0.5\linewidth}{\textbf{%
% Descomente as op��es que se aplicam ao seu caso
%Proposta de Tema para Qualifica��o}
%Disserta��o de Mestrado}
Relat�rio} apresentado � Disciplina de Administra��o de Sistemas Propriet�rios
como parte dos requisitos para obten��o da nota da 1a unidade}


\vfill

\large

%Data
Natal, RN, junho de 2014

\end{center}

\end{titlepage}
