% newcommand define novos comandos, que podem passar a ser usados da
% mesma forma que os comandos LaTeX de base.

% Configura��o da fonte
%\renewcommand{\familydefault}{\sfdefault}

% Margens ----------------------------------------------------------------------
%\setlength{\oddsidemargin}{3.5cm}
%\setlength{\evensidemargin}{2.5cm}
%\setlength{\textwidth}{15cm}
%\addtolength{\oddsidemargin}{-1in}
%\addtolength{\evensidemargin}{-1in}
%%
%\setlength{\topmargin}{2.0cm}
%\setlength{\headheight}{1.0cm}
%\setlength{\headsep}{1.0cm}
%\setlength{\textheight}{22.7cm}
%\setlength{\footskip}{1.0cm}
%\addtolength{\topmargin}{-1in}

% Cap�tulos --------------------------------------------------------------------
% N�o aparecer o n�mero na primeira p�gina dos cap�tulos
\newcommand{\mychapter}[1]{\chapter{#1}\thispagestyle{empty}}
\newcommand{\mychapterstar}[1]{\chapter*{#1}\thispagestyle{empty}}

% Os cap�tulos sem n�mero
%\newcommand{\mychapterast}[1]{\chapter*{#1}\thispagestyle{empty}
%\chaptermark{#1}
%\afterpage{\markboth{\uppercase{#1}}{\rightmark}}
%\markboth{\uppercase{#1}}{}
%}

%Referencias
%\newcommand{\BibTeX}{\textsc{B\hspace{-0.1em}i\hspace{-0.1em}b\hspace{-0.3em}}\TeX}

% Comandos matem�ticos ---------------------------------------------------------
% Implica��o em f�rmulas
%\newcommand{\implica}{\quad\Rightarrow\quad} %Meio de linha
%\newcommand{\implicafim}{\quad\Rightarrow}   %Fim de linha
%\newcommand{\tende}{\rightarrow}

% Fra��o com parenteses
%\newcommand{\pfrac}[2]{\parent{\frac{#1}{#2}}}

% Transformada de Laplace e transformada Z
%\newcommand{\lapl}{\pounds}
%\newcommand{\transfz}{\mathcal{Z}}

% Sequ�ncias
%\newcommand{\sequencia}[4]{$#1_{#2}$, $#1_{#3}$, \ldots, $#1_{#4}$}

% Se��es sem n�mero
%\newcommand{\mysectionast}[1]{\section*{#1}
%\addcontentsline{toc}{section}{#1}
%\markright{\uppercase{#1}}
%}

% No tabularx, as celulas devem ser centradas verticalmente
\renewcommand{\tabularxcolumn}[1]{m{#1}}

% C�lulas centralizadas horizontalmente no tabularx
\newcolumntype{C}{>{\centering\arraybackslash}X}


% Outros ----------------------------------------------------------------------
%\newcommand{\chave}[1]{\left\{#1\right\}}
%\newcommand{\colchete}[1]{\left[#1\right]}
%\newcommand{\parent}[1]{\left(#1\right)}

%%subsubsection
%\newcounter{subcount}
%\newenvironment{mysub}
%{\begin{list}
%{\textbf{\thesubsubsection.\arabic{subcount}}}
%{\setlength{\itemindent}{2.2em}
%\setlength{\rightmargin}{.6in}
%\setlength{\labelwidth}{1in}
%\setlength{\labelsep}{.2in}
%\setlength{\parsep}{.5ex plus .2ex minus .1ex}
%\setlength{\itemsep}{0ex plus .2ex minus 0ex}
%\usecounter{subcount}}
%}

