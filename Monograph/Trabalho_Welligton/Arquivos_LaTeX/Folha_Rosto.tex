%
% ********** Página de Rosto
%

% titlepage gera páginas sem numeração
\begin{titlepage}

\begin{center}

\small

% O comando @{} no ambiente tabular x é para criar um novo delimitador
% entre colunas que não a barra vertical | que é normalmente utilizada.
% O delimitador desejado vai entre as chaves. No exemplo, não há nada,
% de modo que o delimitador é vazio. Este recurso está sendo usado para
% eliminar o espaço que geralmente existe entre as colunas
\begin{tabularx}{\linewidth}{ c X }
% A figura foi colocada dentro de um parbox para que fique verticalmente
% centralizada em relação ao resto da linha
\parbox[c]{3cm}{\includegraphics[width=\linewidth]{Imagens/inf.jpeg}} &
\begin{center}
\textsf{\textsc{Instituto Federal de Educa‹o Cincia\\
 e Tecnologia do Rio Grande do Norte \\
Diretoria de Gest‹o de Tecnologia da Informa‹o \\
Curso Superior de Tecnologia em Redes de Computadores
}} 
\end{center}

\end{tabularx}


% O vfill é um espaço vertical que assume a máxima dimensão possível
% Os vfill's desta página foram utilizados para que o texto ocupe
% toda a folha
\vfill

\LARGE

\textbf{Aqui Ž o t’tulo de seu trabalho}

\vfill

\Large

\textbf{Fulano do Anz—is Pereira}

\vfill

\normalsize

Prof. Dr. Rodrigo Siqueira Martins

\vfill

\hfill
\parbox{0.5\linewidth}{\textbf{%
% Descomente as opções que se aplicam ao seu caso
%Proposta de Tema para Qualificação}
%Dissertação de Mestrado}
Relat—rio} apresentado ˆ Disciplina de Administra‹o de Sistemas Propriet‡rios
como parte dos requisitos para obten‹o da nota da 1a unidade}


\vfill

\large

%Data
Natal, RN, junho de 2014

\end{center}

\end{titlepage}
